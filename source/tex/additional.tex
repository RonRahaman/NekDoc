
\section{Adaptive Lagrangian-Eulerian (ALE)}

We consider unsteady incompressible flow in a domain with moving boundaries:
\begin{eqnarray} \label{eq:mhd1}
\frac{\partial\mathbf u}{\partial t}&=&-\nabla p +\frac{1}{Re}\nabla\cdot(\nabla + \nabla^T)\mathbf u  + NL,\\
 \nabla \cdot \mathbf u &= &0 
\end{eqnarray}
Here, \(NL\) represents the quadratic nonlinearities from the convective term.

Our free-surface hydrodynamic formulation is based upon the arbitrary 
Lagrangian-Eulerian (ALE) formulation described in \cite{ho89}.
Here, the domain \(\Omega(t)\) is also an unknown.  As with the velocity,
the geometry \(\vect x\) is represented by high-order polynomials.
For viscous free-surface flows,
the rapid convergence of the high-order surface approximation to the 
physically smooth solution minimizes surface-tension-induced stresses
arising from non-physical cusps at the element interfaces, 
where only \(C^0\) continuity is enforced.  
The geometric deformation is specified by a mesh velocity \(\vect w := \dot{\vect x}\)
that is essentially arbitrary, provided that \(\vect w\) satisfies the kinematic
condition \(\vect w \cdot \hat{\vect n}|^{}_{\Gamma} = \vect u \cdot \hat{\vect n}|^{}_{\Gamma}\),
where \(\hat{\vect n}\) is the unit normal at the free surface \(\Gamma(x,y,t)\).
The ALE formulation provides a very accurate description of the free
surface and is appropriate in situations where wave-breaking does not occur.

To highlight the key aspects of the ALE formulation, we introduce
the weighted residual formulation of (\ref{eq:mhd1}):
{\em Find \((\bu,p) \in X^N \times Y^N\) such that:}
\begin{eqnarray} \label{eq:wrt1}
\dd{}{t} (\vect v,\vect u) = (\nabla \cdot \vect v,p) - \frac{2}{Re}(\nabla \vect v,\vect S)
+(\vect v,N\!L) + c(\vect v,\vect w,\vect u),
\qquad
(\nabla \cdot \vect u,q) = 0,
\end{eqnarray} 
for all test functions \((\vect v,q) \in X^N \times Y^N\).
Here \((X^N,Y^N)\) are the compatible velocity-pressure approximation 
spaces introduced in \cite{mapa89}, 
\((.,.)\) denotes the inner-product
\((\vect f,\vect g) := \int_{\Omega(t)} \vect f \cdot \vect g \,dV\),
and 
\(\vect S\) is the stress tensor 
\(S_{ij}^{} := \frac{1}{2}( \pp{u_i}{x_j} + \pp{u_j}{x_i} )\).
For simplicity, we have neglected the surface tension term.
A new term in (\ref{eq:wrt1}) is the trilinear form
involving the mesh velocity
\begin{eqnarray} \label{eq:trilin}
c(\vect v,\vect w,\vect u) :=
\int_{\Omega(t)}^{}
\sum_{i=1}^3 
\sum_{j=1}^3 v_i^{} \pp{w_j^{} u_i^{}}{x_j^{}} \,dV,
\end{eqnarray} 
which derives from the Reynolds transport theorem when
the time derivative is moved outside the bilinear form \((\vect v,\vect u_t^{})\).
The advantage of (\ref{eq:wrt1}) is that it greatly simplifies the
time differencing and avoids grid-to-grid interpolation as the domain
evolves in time.  With the time derivative outside of the integral, 
each bilinear or trilinear form involves functions at a specific time,
\(t^{n-q}\), integrated over \(\Omega(t^{n-q})\).
For example, with a second-order backward-difference/extrapolation scheme,
the discrete form of (\ref{eq:wrt1}) is
\begin{eqnarray} \label{eq:bdk}
\frac{1}{2 \dt}\left[ 
 3 (\vect v^n,\vect u^n)^n
-4 (\vect v^{n-1},\vect u^{n-1})^{n-1}
 + (\vect v^{n-2},\vect u^{n-2})^{n-2} \right]
= L^n (\vect u) + 
2 \widetilde{N\!L}^{n-1}
- \widetilde{N\!L}^{n-2}.
\end{eqnarray} 
Here, \(L^n(\vect u)\) accounts for all {\em linear} terms in (\ref{eq:wrt1}),
including the pressure and divergence-free constraint, which are evaluated
implicitly (i.e., at time level \(t^n\), on \(\Omega(t^n)\)), and
\(\widetilde{N\!L}^{n-q}\) accounts for all {\em nonlinear} terms, including
the mesh motion term (\ref{eq:trilin}), at time-level \(t^{n-q}\).
The superscript on the inner-products \((.,.)^{n-q}\) indicates 
integration over \(\Omega(t^{n-q})\). 
The overall time advancement is as follows.  
The mesh position \(\vect x^n \in \Omega(t^n)\) is computed
explicitly using \(\vect w^{n-1}\) and \(\vect w^{n-2}\);
the new mass, stiffness, and gradient operators involving integrals
and derivatives on \(\Omega(t^n)\) are computed;  
the extrapolated right-hand-side terms are evaluated; and 
the implicit linear system is solved for \(\vect u^n\).   
Note that it is only the {\em operators} that are updated,
not the {\em matrices}.  Matrices are never formed in Nek5000 and
because of this, the overhead for the moving domain formulation
is very low.

